\documentclass[12pt, letterpaper]{article}

\title{Relatorio do Trabalho Individual 3: PGP}
\author{Bernardo Ferrari Mendonça}

\begin{document}
\maketitle

\section{Criar um certificado PGP:}

Foi feito o backup da chave privada em formato \textit{armored} em um disco separado.

A chave pública foi publicado em \textbf{keyserver.ubuntu.com}. Sua chave pública é:
\begin{quote}
  A9263FA40C1B31AD225E63D1F10C65B5802ADBBE
\end{quote}

\section{Criar um novo certificado PGP e revogá-lo:}

Foi criado um novo certificado PGP, sua chave pública é:

\begin{quote}
  AC13A37374425750A38E1453C0D3F333DE42B310
\end{quote}

O certificado foi publicado em \textbf{keyserver.ubuntu.com}
\begin{quote}
  gpg --send-keys --keyserver keyserver.ubuntu.com AC13A373
\end{quote}
logo após, foi criado um certificado de revogação utilizando o comando:

\begin{quote}
  gpg --gen-revoke AC13A373 \textgreater{} revoke.asc
\end{quote}

O certificado de revogação foi salvo em um arquivo \textbf{revoke.asc}
e foi e importado para a \textit{keyring}:
\begin{quote}
  gpg --import revoke.asc
\end{quote}
então, é enviada a chave já revogada novamente para o servidor de chaves:
\begin{quote}
  gpg --send-keys --keyserver keyserver.ubuntu.com AC13A373
\end{quote}
para testar se a chave foi atualizada no \textit{keyserver},
foi importada a chave em um segundo computador onde foi obtido o seguinte resultado:

\begin{quote}
\begin{verbatim}
pub   rsa2048 2019-10-03 [SC] [revoked: 2019-10-03]
      AC13A37374425750A38E1453C0D3F333DE42B310
uid   [ revoked] Bernardo Ferrari (quick test key)
<bernardom.ferrari@gmail.com>
\end{verbatim}
\end{quote}

\section{Assine o certificado PGP de outra pessoa,
depois revogue essa assinatura:}

Primeiro foi importada a chave publica do Francisco Vicenzi:
\begin{quote}
  gpg --keyserver keyserver.ubuntu.com --recv-keys C0153CE4
\end{quote}
Depois foi a chave foi editada rodando os seguintes comandos:
\begin{quote}
  gpg gpg --edit-key franciscovicenzi@gmail.com \\
  gpg \textgreater{} sign \\
  gpg \textgreater{} save
\end{quote}
Depois for enviada a chave para o \textit{keyserver}:
\begin{quote}
  gpg --keyserver keyserver.ubuntu.com --send-keys C0153CE4
\end{quote}
Depois de enviado, a assinatura apareceu no \textit{keyserver}:
\begin{quote}
\begin{verbatim}
sig  sig  f10c65b5802adbbe 2019-10-08t19:12:13z
____________________ ____________________ f10c65b5802adbbe
\end{verbatim}
\end{quote}

Após a assinatura, foi feita a revogação da mesma:
\begin{quote}
  gpg gpg --edit-key franciscovicenzi@gmail.com \\
  gpg \textgreater{} revsig \\
  gpg \textgreater{} save
\end{quote}
Depois foi enviada a chave revogada para o \textit{keyserver}:
\begin{quote}
  gpg --keyserver keyserver.ubuntu.com --send-keys C0153CE4
\end{quote}
Depois de enviada, foi feita a verificação no \textit{keyserver}:
\begin{quote}
\begin{verbatim}
sig  sig  f10c65b5802adbbe 2019-10-08t19:12:13z
____________________ ____________________ f10c65b5802adbbe
sig revok f10c65b5802adbbe 2019-10-08T19:40:34Z
____________________ ____________________ f10c65b5802adbbe
\end{verbatim}
\end{quote}

\section{O que é o anel de chaves privadas:}

O \textit{keyring}, ou Anel de chaves, é separado em
um \textit{keyring} de chaves públicas e um \textit{keyring}
de chaves privadas.

O \textit{keyring} de chaves públicas é armazenado
em um arquivo chamado \textbf{pubring.kbx} e contém
todas as informações de cada chave pública importada
no \textit{keyring}.

O \textit{keyring} de chaves privadas é armazenado
em um diretório chamado \textbf{private-keys-v1.d}
e dentro cada chave privada é armazenada em seu
próprio arquivo identificado por um número chamado
\textbf{keygrip}.

Todas essas estruturas ficam guardadas dentro de
um diretório chamado:
\begin{quote}
  \textbf{\$HOME/.gnupg/}
\end{quote}
Este diretório é construido com permições de \textbf{leitura}, \textbf{escrita}
e \textbf{execução} somente para o usuário dono da \textit{keyring}.

\section{Diferença entre assinar uma chave local e assinar em um servidor:}

Uma chave pública assinada cria uma relação de confiabilidade entre
a chave assinante e a chave assinada, é vantajoso para o portador de
uma chave pública que sua chave estaja assinada por varias chaves distintas.

Assinatura de chaves são feitas para garantir duas propriedades principais:
\begin{itemize}
    \item ela permite a detecção de fraudes e permite que você certifique
      que uma dada chave realmente pertence a pessoa referente ao \textit{UserID}
      da chave.
    \item ela faz parte de uma rede de confiabilidade chamada de \textit{web of trust}
      para extender a confiabilidade de maneira transitiva entre assinaturas. Esse conseito
      será explorado a fundo na seção~\ref{weboftrust}.
\end{itemize}
Usuarios responsaveis que seguem boas praticas de assinatura de chaves conseguem
detectar e se proteger de qualquer eventual fraude de identidade.

Idealmente, você distribuiria sua chave pessoalmente entragando cópias para seus correspondentes.
Porém, na pratica, chaves acabam sendo distribuidas por email ou algum outro meio de comunicação.
A distribuição por email é uma boa pratica quando você tem poucos correspondentes, ou até mesmo
se você possui muitos correspondentes. Pode-se usar como uma alternativa publicar sua chave
publica em um site pessoal na internet. Porém, isso se torna um problema quando um terceiro que
deseja se comunicar não sabe onde encontrar sua chave.

Para resolver esse problema, \textit{keyservers} ou servidores de chaves publicas são
usados para coletar e distribuir chaves públicas.
Uma chave pública recebiada por um \textit{keyserver} é:
\begin{itemize}
  \item addicionada ao \textit{keyring} de chaves públicas; ou,
  \item é feita uma fusão com a chave existente no servidor.
\end{itemize}
Quando alguem precisar de sua chave pública, essa pessoa consulta sua chave na sua \textit{keyring}
e responde a consulta com a verção da sua chave publica que se encontra atualizada no \textit{keyserver}.

O uso de um \textit{keyserver} é vantajoso quando varias pessoas estão frequentemente validando
as chaves de outras pessoas. Sem o uso de um \textit{keyserver}, quando Bob assina a chave de Alice,
Bob deveria enviar para Alice uma copia da chave publica de Alice assinada por Bob para que Alice
atualize sua propria chave publica e que possa re-distribuir sua nova chave pública assinada por Bob
para seus correspondentes. Esse esforço é necessario para que Alice e Bob cumpram seu papel em construir
uma rede de confiabilidade, assim contribuindo com a segurança do protocolo PGP\@.

Utilizar um \textit{keyserver} simplifica o processo de assinatura de chaves.
Quando bob assina a chave de Alice, ele envia essa chave assinada para o \textit{keyserver}.
Então o servidor adiciona a assinatura de Bob a sua cópia da chave de Alice.
Terceiros que desejam atualizar suas cópias da chave de Alice podem consultar o servidor
por conta própria. Assim Alice não fica envolvida no processo de districuição da sua chave publica
pois o \textit{keyserver} toma esse papel.

\section{O que é a \textit{web of trust} e como ela é organizada:}\label{weboftrust}

A \textit{web of trust} ou, banco de dados de confiabilidade,
é organizada a partir de assinaturas de confiabilidade.
Se Alice confia na chave pública de Bob, Alice importa
a chave pública de Bob para sua \textit{keyring} e
assina a chava publica de Bob utilizando sua chave privada.

Essa Assinatura pode tomar os seguintes valores de confiabilidade,
também chamados de \textit{trust values}:
\begin{itemize}
  \item \textbf{unknown:}
    Não se sabe nada sobre a capacidade do dono da chave quanto a segurança de assinaturas.
    Chaves na sua \textit{keyring} pública que você não é dono inicialmente possuem esse \textit{trust level}.
  \item \textbf{never:}
    É de seu conhecimento que o dono da chave utiliza as assinaturas de forma imprópria.
  \item \textbf{marginal:}
    O dono da chave entende as implicações de assinaturas e valida as chaves antes de assinalas.
  \item \textbf{full:}
    O dono da chave é excepcional quanto a assinatura de chaves.
  \item \textbf{ultimate:}
    Valor especial para chaves do dono da \textit{keyring}.
\end{itemize}

A chave publica assinada pode ser devolvida para o dono da chave ou publicada em um
\textit{keyserver} para que todos saibam que você confica na chave publica.

No protocolo de \textit{web of trust} do PGP, um algoritmo mais flexivel de confiabilidade pode ser utilizado:
uma chave \textbf{K} pode ser considerada valida se ela:
\begin{enumerate}
    \item é assinada por uma quantidade suficiente de chaves, ou seja:
      \begin{enumerate}
          \item você mesmo assinou essa chave, ou;
          \item foi assinada por uma chave de confiabilidade \textit{full}, ou;
          \item foi assinada por três chaves de confiabilidade \textit{marginal}, e;
      \end{enumerate}
    \item o caminho entre a chave \textbf{K} até sua chave é de
      cinco passos ou menos.
\end{enumerate}

Seguindo esse algoritmo é consedido a cada chave da keyring um valor de validade possivelmente
diferente do \textit{trust value}.

\section{O que são sub-chaves e pra que elas servem?}

O protocolo PGP da suporte a sub-chaves, ou \textit{subkeys}. \textit{Subkeys} são iguais a
pares de chaves normais exceto pelo fato de estarem ligadas a um par de chaves mestre.
O GPG cria um par de chaves mestre cujo a chave mestre prívada é somente para assinaturas
e cria uma \textit{subkey} privada somente para criptografar.

As \textit{subkeys} facilitam o gerenciamento de chaves. A chave mestre é muito importante
pois é a melhor prova de sua identidade online para vários serviços. Na eventual perda da
chave mestre privada, será necessario construir a \textit{web of trust} novamente.
É necessario manter a chave mestre privada super segura. Contudo, manter todas suas chaves
de mandaira extremamente segura é inconveniente.

Usando \textit{subkeys} é possivel manter somente as subkeys nos computadores de uso pessoal
e guardar a chave privada em segurança.

Após criar uma segunda \textit{subkey} para assinaturas, só sera necessario o uso da chave privada
nas seguintes ocasiões:
\begin{itemize}
  \item quando você assinal ou revogar a chave publica de outra pessoa; ou
  \item quando você addicionar um novo UID ou editar um existente; ou
  \item quando você criar uma nova \textit{subkey}; ou
  \item quando você revogar um UID ou \textit{subkey}; ou
  \item quando você editar a data de vencimento da chave mestre ou qualquer \textit{subkey}; ou
  \item quando você revogar ou gerar um certificado de revogação para uma chave completa.
\end{itemize}

\section{Coloque uma foto no seu certificado GPG}

Para colocar uma foto em meu certificado GPG foram executados os seguintes comandos:
\begin{quote}
  gpg --edit-key bernardo.mferrari@gmail.com \\
  gpg \textgreater{} addphoto pikachu.jpg \\
  gpg \textgreater{} save
\end{quote}

\section{O que é preciso para criar e manter um servidor de chaves GPG, sincronizado com os demais servidores existentes?}

No protocolo PGP clientes acabam confiando em \textit{keyservers} PGP para guardar e receber copias das chaves suas chaves
publicas e das chaves publicas de terceiros. Essa comunicação entre os clientes PGP e \textit{keyservers} PGP se da utilizando
o protocolo HKP (OpenPGP HTTP Keyserver Protocol) ou HKPS para a verção segura TLS\@.

Um \textit{keyserver} malicioso poderia alterar as chaves e manipular a \textit{\textbf{web of trust}} ao seu favor, para isso
foram criadas as \textit{\textbf{keyserver pool}}.

\subsection{A \textit{keyserver pool}:}

Para uma melhor segurança os \textit{keyservers} são organizados em um \textbf{DNS}.
Quando um cliente OpenPGP precisa se comunicar com um \textit{keyservers} ele consulta
o endereço hkps.pool.sks-keyservers.net que retorna a tablela de endereços ipv4 e ipv6
de \textit{keyservers} onde o cliente escolhera aleatoriamente com quem se conectar.

O dominio sks-keyservers.net é segurado pelo DNSSEC\@.

\subsection{Como criar seu proprio \textit{keyserver} e conectalo à \textit{keyserver pool}: }

Para configurar seu proprio OpenPGP \textit{keyserver} é utilizado o \textit{\textbf{Synchronizing OpenPGP Key Server (SKS)}}.
Esse servidor vai ser inserido em uma \textit{keyserver pool} e se manter atualizado por uma \textit{mailing list} SKS do OpenPGP\@.

\section{Dê um exemplo de como tornar sigiloso um arquivo usando o GPG:}

Suponha que desejamos enviar um arquivo de texto \textbf{mydoc.txt} para outra pessoa, porém,
precisamos tornar esse texto sigiloso para que nenhum terceiro possa compreender seu conteudo.
O protocolo PGP torna isso possivel através dos seguintes comandos:
\begin{quote}
  gpg --armor --output mydoc.crypt.txt --encrypt --recipient nicolas0pfeifer@gmail.com mydoc.txt
\end{quote}
Assim o arquivo \textbf{mydoc.crypt.txt} só pode ser descriptografado pelo portador do par privado da chave pública utilizada
para cifragem, o par foi identificado pelo \textit{UserID} nicolas0pfeifer@gmail.com.

Foi me enviada também uma mensagem \textbf{oi.cript} criptografada com minha chave publica.
Para descriptografar esta mensagem cifrada com a minha chave privada é utilizado o seguinte comando:
\begin{quote}
  gpg --output oi --decrypt oi.cript
\end{quote}

\section{Mostre um exemplo de como assinar um arquivo (assinatura anexada e outro com assinatura separada), usando o GPG:}

\end{document}
