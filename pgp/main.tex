\documentclass[12pt, letterpaper]{article}

\title{Relatorio do Trabalho Individual 3: PGP}
\author{Bernardo Ferrari Mendonça}

\begin{document}
\maketitle

\section{Criar um certificado PGP:}

Foi feito o backup da chave privada em formato \textit{armored} em um disco separado.

A chave privada foi publicado em \textbf{keyserver.ubuntu.com}.

\textbf{TODO:} Dizer qual é a chave.

\section{Criar um novo certificado PGP e revogá-lo:}

Foi criado um novo certificado PGP, sua chave pública é:

\begin{quote}
  AC13A37374425750A38E1453C0D3F333DE42B310
\end{quote}

O certificado foi publicado em \textbf{keyserver.ubuntu.com}
\begin{quote}
  gpg --send-keys --keyserver keyserver.ubuntu.com AC13A373
\end{quote}
logo após, foi criado um certificado de revogação utilizando o comando:

\begin{quote}
  gpg --gen-revoke AC13A373 \textgreater{} revoke.asc
\end{quote}

O certificado de revogação foi salvo em um arquivo \textbf{revoke.asc}
e foi e importado para a \textit{keyring}:
\begin{quote}
  gpg --import revoke.asc
\end{quote}
então, é enviada a chave já revogada novamente para o servidor de chaves:
\begin{quote}
  gpg --send-keys --keyserver keyserver.ubuntu.com AC13A373
\end{quote}
para testar se a chave foi atualizada no \textit{keyserver},
foi importada a chave em um segundo computador onde foi obtido o seguinte resultado:

\begin{quote}
\begin{verbatim}
pub   rsa2048 2019-10-03 [SC] [revoked: 2019-10-03]
      AC13A37374425750A38E1453C0D3F333DE42B310
uid   [ revoked] Bernardo Ferrari (quick test key)
<bernardom.ferrari@gmail.com>
\end{verbatim}
\end{quote}

\section{Assine o certificado PGP de outra pessoa, depois revogue essa assinatura:}

\end{document}
