\documentclass[12pt, letterpaper]{article}

\title{Relatorio do Trabalho Individual 3: PGP}
\author{Bernardo Ferrari Mendonça}

\begin{document}
\maketitle

\section{Criar um certificado PGP:}

Foi feito o backup da chave privada em formato \textit{armored} em um disco separado.

A chave privada foi publicado em \textbf{keyserver.ubuntu.com}.

\textbf{TODO:} Dizer qual é a chave.

\section{Criar um novo certificado PGP e revogá-lo:}

Foi criado um novo certificado PGP, sua chave pública é:

\begin{quote}
  AC13A37374425750A38E1453C0D3F333DE42B310
\end{quote}

O certificado foi publicado em \textbf{keyserver.ubuntu.com}
\begin{quote}
  gpg --send-keys --keyserver keyserver.ubuntu.com AC13A373
\end{quote}
logo após, foi criado um certificado de revogação utilizando o comando:

\begin{quote}
  gpg --gen-revoke AC13A373 \textgreater{} revoke.asc
\end{quote}

O certificado de revogação foi salvo em um arquivo \textbf{revoke.asc}
e foi e importado para a \textit{keyring}:
\begin{quote}
  gpg --import revoke.asc
\end{quote}
então, é enviada a chave já revogada novamente para o servidor de chaves:
\begin{quote}
  gpg --send-keys --keyserver keyserver.ubuntu.com AC13A373
\end{quote}
para testar se a chave foi atualizada no \textit{keyserver},
foi importada a chave em um segundo computador onde foi obtido o seguinte resultado:

\begin{quote}
\begin{verbatim}
pub   rsa2048 2019-10-03 [SC] [revoked: 2019-10-03]
      AC13A37374425750A38E1453C0D3F333DE42B310
uid   [ revoked] Bernardo Ferrari (quick test key)
<bernardom.ferrari@gmail.com>
\end{verbatim}
\end{quote}

\section{Assine o certificado PGP de outra pessoa,
depois revogue essa assinatura:}

\textbf{TODO:} Fazer a chave e assinar.

\section{O que é o anel de chaves privadas:}

O \textit{keyring}, ou Anel de chaves, é separado em
um \textit{keyring} de chaves públicas e um \textit{keyring}
de chaves privadas.

O \textit{keyring} de chaves públicas é armazenado
em um arquivo chamado \textbf{pubring.kbx} e contém
todas as informações de cada chave pública importada
no \textit{keyring}.

O \textit{keyring} de chaves privadas é armazenado
em um diretório chamado \textbf{private-keys-v1.d}
e dentro cada chave privada é armazenada em seu
próprio arquivo identificado por um número chamado
\textbf{keygrip}.

Todas essas estruturas ficam guardadas dentro de
um diretório chamado:
\begin{quote}
  \textbf{\$HOME/.gnupg/}
\end{quote}
Este diretório é construido com permições de \textbf{leitura}, \textbf{escrita}
e \textbf{execução} somente para o usuário dono da \textit{keyring}.

\section{Diferença entre assinar uma chave local e assinar em um servidor:}

\textbf{TODO:} Assinar no servidor seria dar push da chave assinada no servidor?

\section{O que é a \textit{web of trust} e como ela é organizada:}

A \textit{web of trust} ou, banco de dados de confiabilidade,
é organizada a partir de assinaturas de confiabilidade.
Se Alice confia na chave pública de Bob, Alice importa
a chave pública de Bob para sua \textit{keyring} e
assina a chava publica de Bob utilizando sua chave privada.

Essa Assinatura pode tomar os seguintes valores de confiabilidade,
também chamados de \textit{trust values}:
\begin{itemize}
  \item \textbf{unknown:}
    Não se sabe nada sobre a capacidade do dono da chave quanto a segurança de assinaturas.
    Chaves na sua \textit{keyring} pública que você não é dono inicialmente possuem esse \textit{trust level}.
  \item \textbf{never:}
    É de seu conhecimento que o dono da chave utiliza as assinaturas de forma imprópria.
  \item \textbf{marginal:}
    O dono da chave entende as implicações de assinaturas e valida as chaves antes de assinalas.
  \item \textbf{full:}
    O dono da chave é excepcional quanto a assinatura de chaves.
  \item \textbf{ultimate:}
    Valor especial para chaves do dono da \textit{keyring}.
\end{itemize}

A chave publica assinada pode ser devolvida para o dono da chave ou publicada em um
\textit{keyserver} para que todos saibam que você confica na chave publica.

No protocolo de \textit{web of trust} do PGP, um algoritmo mais flexivel de confiabilidade pode ser utilizado:
uma chave \textbf{K} pode ser considerada valida se ela:
\begin{enumerate}
    \item é assinada por uma quantidade suficiente de chaves, ou seja:
      \begin{enumerate}
          \item você mesmo assinou essa chave, ou;
          \item foi assinada por uma chave de confiabilidade \textit{full}, ou;
          \item foi assinada por três chaves de confiabilidade \textit{marginal}, e;
      \end{enumerate}
    \item o caminho entre a chave \textbf{K} até sua chave é de
      cinco passos ou menos.
\end{enumerate}

Seguindo esse algoritmo é consedido a cada chave da keyring um valor de validade possivelmente
diferente do \textit{trust value}.

\end{document}
